\documentclass[11pt,svgnames,smaller]{beamer}

\usepackage{CommandsAndStyle}

\usepackage{listings}
\usepackage{amsthm}
\usepackage{amsmath}
\usepackage{subfig}
\usepackage{xcolor}
\usepackage{textpos}
\usepackage{transparent}
\usepackage{tikz}
\usepackage{epigraph}


\newcommand{\fsamp}{\textsc{F-samp}\xspace}
\newcommand{\base}{\textsc{base}\xspace}
\newcommand{\fcount}{\textsc{F-count}\xspace}
\mode<presentation>

\author{Gaspare Ferraro}

\author[Gaspare Ferraro]{\includegraphics[height=2cm]{images/cherubino}\\Gaspare Ferraro\\ \vspace{10pt} \small{Relatori\\ Roberto Grossi, Andrea Marino}}
\institute[Università di Pisa]{Università di Pisa\\Dipartimento di Informatica}

\setbeamertemplate{section page}
{
	\begin{centering}
		\begin{beamercolorbox}[sep=16pt,center]{part title}
			\usebeamerfont{section title}\insertsection\par
		\end{beamercolorbox}
	\end{centering}
}

\title{Similarità di sottografi nelle reti complesse}
\date{Pisa, 1 dicembre 2017}
\titlegraphic{\vfill}
% \includegraphics[height=0.4cm]{images/creative_commons.png}}

\setbeamercolor{title}{fg=black!65!black}

% \author[Relatore]{Roberto Grossi}

\begin{document}
	
	\begin{frame} 
	\titlepage
	\end{frame}
				
	\logo{\transparent{0.2}\includegraphics[height=2cm]{images/cherubino}}

	\part{Il problema}

\begin{frame}
	\partpage
	\centering
\end{frame}

\begin{frame}
	\frametitle{Reti complesse}
	\centering
	\begin{figure}[h]
		\centering
		\begin{flushleft}
			Grafi con caratteristiche topologiche non banali che occorrono modellando 
			
			sistemi reali (quali social network, reti neurali, computer network, ...).
		\end{flushleft}
		\medskip
		
		\pause
		\small 
		\begin{minipage}[t]{.45\textwidth}
			\centering
			\includegraphics[width=1\textwidth]{images/9_social}
			\small 
			\caption{Cluster di amicizie in un social network } % \\ \textit{Fonte: SNAP Stanford}}
		\end{minipage}\hfill
		\pause
		\begin{minipage}[t]{.45\textwidth}
			\centering
			\includegraphics[width=1\textwidth]{images/7_flight}
			\small 
			\caption{Rotte dei voli commerciali} % \\ \textit{Fonte: Bio Diaspora, Toronto}}
		\end{minipage}
	\end{figure}
\end{frame}

\begin{frame}
	\frametitle{Indici di similarità}
	\centering
	
	\small
	\pause
	\begin{figure}[h]
		\begin{minipage}[t]{.48\textwidth}
			\centering
			\Large
			Jaccard
			\small
			\medskip
			\begin{equation*}
				J(A,B) = \frac{|A \cap B|}{|A \cup B|}
			\end{equation*}
			\includegraphics[width=0.5\textwidth]{images/4_jaccard}
		\end{minipage}\hfill
		\pause
		\begin{minipage}[t]{.48\textwidth}
			\centering
			\Large
			Bray-Curtis
			\small
			\medskip
			\begin{equation*}
			BC(A,B) = \frac{2 \times |A \cap B|}{|A| + |B|}
			\end{equation*}
			\includegraphics[width=0.6\textwidth]{images/5_bray_curtis}
			
		\end{minipage}\hfill		
	\end{figure}
	\small
	\pause
	
	$J(A,B) = BC(A,B) = 0$ se $A \cap B = \emptyset$\medskip
	
	$J(A,B) = BC(A,B) = 1$ se $A = B$ \phantom{$\cap \emptyset.$}
	
\end{frame}

\begin{frame}
	\frametitle{Reti etichettate e $q$-grammi}
	
	\textit{"Nessun uomo è un'isola, completo in se stesso; ogni uomo è un pezzo del continente, una parte del tutto."}
	\begin{flushright}
		\small \textit{John Donne}
	\end{flushright}
	
	\centering
	\textit{Analizzare non il solo nodo, ma anche la sua interfaccia verso l'esterno!}\\
	
	\pause
	
	Come modellare le interazioni?\\
	
	\pause
	
	\begin{figure}[h]
		\centering
		\begin{minipage}[t]{.49\textwidth}
			\centering
			Rete etichettata\medskip
			
			\includegraphics[width=0.5\textwidth]{images/11_labeled}
		\end{minipage}\hfill
		\pause
		\begin{minipage}[t]{.49\textwidth}
			
			\centering
			
			\textbf{q-grammi}: sottosequenza di $q$ elementi consecutivi in un testo
			
			+
			
			\textbf{q-path}: cammino di $q$ nodi \textit{distinti} collegati in un grafo \bigskip
			
			\small\pause
			
			\textbf{Esempio} $3$-grammi che terminano in $0$:
			\begin{itemize}
				\item $(2-1-0)$: \color{red}cba \color{black}
				\item $(4-3-0)$: \color{green}bca \color{black}
				\item $(6-5-0)$: \color{violet}aca \color{black}
				\item $(8-7-0)$: \color{orange}baa \color{black}
			\end{itemize}
		\end{minipage}
	\end{figure}

	
\end{frame}


\begin{frame}
	\frametitle{Frequenze dei $q$-grammi}
	
	\textbf{Notazione:}  
	\small \center	
	$f_X[w] = y \rightarrow$ Il $q$-gramma $w$ ha frequenza $y$ nei $q$-path che terminano in nodi di $X$\medskip
	
	\pause
	
	\textbf{Esempio:}  
	
	\includegraphics[width=0.4\textwidth]{images/12_freq}
		
	Dato $X = \{0, 1\}$ e $q=3$ abbiamo:
	\centering
	\begin{itemize}
		\item $f_X[bba] = 3$ (path: 4-2-1, 6-5-0, 10-9-1).
		\item $f_X[bca] = 1$ (path: 8-7-0).
		\item $f_X[cba] = 2$ (path: 3-2-1, 11-9-1).
	\end{itemize}
\end{frame}

\begin{frame}
	\frametitle{Il problema}
	
	\begin{flushleft}
		Dato un grafo $G=(V,E,L)$, etichettato su un alfabeto $\Sigma$, ed un intero $q$,
		calcolare la similarità tra due porzioni di grafo $A, B \subset V$ in base alle frequenze
		dei $q$-grammi dei $q$-path che terminano in nodi di $A$ e $B$.
	\end{flushleft}

	\pause
			
	Estendiamo i due indici ai $q$-grammi:

	\begin{equation*}\label{jaccard-sub}	
		J(A,B) = \frac{|A \cap B|}{|A \cup B|} \implies J(A,B) = \frac{ \Sigma_{x \in \Sigma^{q}} \min(f_{A}[x], f_{B}[x]) }{ \Sigma_{x \in \Sigma^{q}} f_{A \cup B}[x] }
	\end{equation*}

	\begin{equation*}\label{bray-sub}
		BC(A,B) = \frac{2 \times |A \cap B|}{|A| + |B|} \implies BC(A,B) = \frac{ 2 \times \Sigma_{x \in \Sigma^{q}} \min(f_{A}[x], f_{B}[x]) }{ \Sigma_{x \in \Sigma^{q}} (f_{A}[x] + f_{B}[x]) }
	\end{equation*}
	
\end{frame}

\begin{frame}
	\frametitle{Applicazioni pratiche}
	
	\pause
	\centering
	\begin{figure}[h]
		\centering
		\begin{minipage}[t]{.49\textwidth}
			\centering
			\textbf{NetInf}\medskip
			
			\includegraphics[width=0.9\textwidth]{images/4_netinf}
			\caption{Diffusione delle notizie tra i vari blog e siti di informazione statunitensi\\ \textit{Fonte: SNAP Stanford}}
		\end{minipage}\hfill
		\pause
		\begin{minipage}[t]{.49\textwidth}
			\centering
			\textbf{IMDb}\medskip
			
			\includegraphics[width=0.9\textwidth]{images/6_imdb}
			\caption{Interazione tra i film con attori in comune\\ \textit{Fonte: IMDb}}
		\end{minipage}
	\end{figure}
\end{frame}

	\part{Approcci di risoluzione}

\begin{frame}
	\partpage
	\centering
\end{frame}

\begin{frame}
	\frametitle{Ricerca esaustiva}
	
	Approccio \textbf{Brute-force}:
	\begin{itemize}
		\item Enumerare tutti i $q$-path esistenti
		\item Contare le frequenze esatte di $f_A[w]$ e $f_B[w]$
		\item Calcolare la similarità usando la definizione
	\end{itemize}

	\medskip
	
	Complessità:
	\begin{itemize}
		\item Tempo: $O(|V|^q)$
		\item Spazio: $O(|\Sigma|^q\ q)$
	\end{itemize}

	\centering
	\medskip
	\textbf{Problema!}
	
	Limitare la ricerca mantenendo inalterato il valore di similarità
	
	\begin{itemize}
		\item Tempo: $O(|V|^q)$ $\rightarrow$ Color Coding $\rightarrow$ $O(2^q\ |V|)$
		\item Spazio: $O(|\Sigma|^q\ q)$ $\rightarrow$ Sampling $\rightarrow$ $O(rq)$
	\end{itemize}
	
	
		
\end{frame}

\begin{frame}
	\frametitle{Color Coding}
	\centering


		\begin{minipage}{.45\textwidth}
			\centering
			Coloriamo casualmente il grafo con $q$ colori e ci limitiamo ai path colorful 
			(percorsi con colori non ripetuti)
			\medskip
			
			\includegraphics[width=0.5\textwidth]{images/8_cc_graph}
			
			\small
			\medskip
			
			Il numero dei path è esponenzialmente ridotto di un fattore $q! / q^q \simeq e^{-q}$\\
			 
			Per $q=3$ solo il $\sim22.22\%$\\
			Per $q=6$ solo il $\sim1.5\%$\phantom{$22$}
		\end{minipage}\hfill
		\begin{minipage}{.45\textwidth}
			\centering
			
			\small
			
			$q!$ colorazioni accettabili\\
			$q^q$ possibili colorazioni
			
			\medskip
			
			\includegraphics[width=0.5\textwidth]{images/8_cc_list}
			
			Esempi di possibili path\medskip
			
			In questo modo:
			\begin{equation*}
				f'_X[w] \simeq e^{-q} f_X[w]
			\end{equation*}
			\hfill
		\end{minipage}\hfill

\end{frame}

\begin{frame}
	\frametitle{Sampling}
	\centering
	
	\begin{minipage}{.45\textwidth}
		\centering
		
		Tabella frequenze dei $q$-grammi: \medskip
		
		\begin{tabular}{|c|c|}
			\hline
			w   & $f_X[w]$  \\ \hline
			aaa &  721 \\ \hline
			abc &  243 \\ \hline
			... & ... \\ \hline
			zzy &   13 \\ \hline
			zzz &   368 \\ \hline
		\end{tabular}
	
		\medskip
		Potenzialmente:
		\medskip		 
		
		$|w| = |\Sigma|^q$ (tutti i $q$-grammi)
		\medskip		 
		 
		$\Sigma_w{f_X[w]} = |V|^q$ (tutti i $q$-path)
		
		
	\end{minipage}\hfill
	\begin{minipage}{.45\textwidth}
		\centering
		Riduciamo le dimensioni della tabella \textbf{campionando uniformemente} $r$ colorful $q$-path distinti.\medskip
		
		Definiamo quindi:
		
		\begin{itemize}
			\item $R$ l'insieme dei $r$ $q$-path campionati ($r \ll |V|^q$ )
			\item $\mathcal{W}$ l'insieme dei $q$-grammi dei $q$-path in $R$ ($|W| \leq r$)
		\end{itemize}
		
		
		
		\hfill
	\end{minipage}\hfill

	\bigskip
	
	\textbf{Jaccard}: campionamento con $X = A \cup B$
	
	\textbf{Bray-Curtis}: campionamento con $X = A \uplus B$
	
	
	
\end{frame}

\begin{frame}
	\frametitle{Esempio di sampling}
	\centering
	
	Campioniamo 5 diversi $3$-path da $X = A \cup B = \{ 0, 1, 12 \}$
	
	\includegraphics[width=0.5\textwidth]{images/13_sampl}
	
	\medskip
	
	R = \{ \color{green} 4-2-1 \color{darkgreen} 10-9-1  \color{green} 8-7-0 \color{darkgreen} 16-13-12  \color{green} 7-14-12 \color{black}\}
	
	$\mathcal{W} = $ \{ acb, bba, bca \}
	
	
\end{frame}

\begin{frame}
\frametitle{Approssimazione degli indici}
\centering

Dato un campione $\mathcal{W}$ di $q$-grammi approssimiamo i due indici limitandoci alle sole stringhe 

\begin{equation*}\label{jaccard-sub}	
J(A,B) = \frac{ \Sigma_{x \in \Sigma^{q}} \min(f_{A}[x], f_{B}[x]) }{ \Sigma_{x \in \Sigma^{q}} f_{A \cup B}[x] }
\end{equation*}
$\downarrow$
\begin{equation*}\label{jaccard-sub}	
J_{\mathcal{W}}(A,B) = \frac{ \Sigma_{x \in \mathcal{W}} \min(f_{A}[x], f_{B}[x]) }{ \Sigma_{x \in \mathcal{W}} f_{A \cup B}[x] }
\end{equation*}

\begin{equation*}\label{bray-sub}
BC(A,B) = \frac{ 2 \times \Sigma_{x \in \Sigma^{q}} \min(f_{A}[x], f_{B}[x]) }{ \Sigma_{x \in \Sigma^{q}} (f_{A}[x] + f_{B}[x]) }
\end{equation*}
$\downarrow$
\begin{equation*}\label{bray-sub}
BC_{\mathcal{W}}(A,B) = \frac{ 2 \times \Sigma_{x \in \mathcal{W}} \min(f_{A}[x], f_{B}[x]) }{ \Sigma_{x \in \mathcal{W}} (f_{A}[x] + f_{B}[x]) }
\end{equation*}

\end{frame}

\begin{frame}
	\frametitle{F-Count e F-Samp}
	\centering
	
	Come calcoliamo $f_A[w]$ e $f_B[w]$ per $w \in \mathcal{W}$?\medskip
	
	\begin{minipage}{.45\textwidth}
		\centering
		\textbf{F-Count}
		
		Calcoliamo in modo esatto i valori di $f_A[w]$ e $f_B[w]$
		con una ricerca esaustiva limitata ai $q$-grammi in $\mathcal{W}$
		
		\bigskip
		
		\small		
		\textbf{Pro}:
		\begin{itemize}
			\item Più preciso in quanto usiamo le frequenze esatte
		\end{itemize}
		
		\textbf{Contro}:
		\begin{itemize}
			\item Potenzialmente lento in quanto potrebbe analizzare una grande porzione di grafo
		\end{itemize}
		
		\hfill
	\end{minipage}\hfill
	\begin{minipage}{.45\textwidth}
		\centering
		\textbf{F-Samp}	
		
		 Stimiamo i valori di $f_A[w]$ e $f_B[w]$ 
		 usando il campione dei $q$-path $R$
		 
		 
		 \bigskip
		 
		 \small		
		 \textbf{Pro}:
		 \begin{itemize}
		 	\item Più veloce poichè analizziamo solo gli $r$ $q$-path campionati
		 \end{itemize}
		 
		 \textbf{Contro}:
		 \begin{itemize}
		 	\item Stima meno precisa dato che usiamo valori approssimati delle frequenze
		 \end{itemize}
		 
		\hfill
	\end{minipage}\hfill
	
\end{frame}

%\begin{frame}
%	\frametitle{Baseline}
%	\centering
%\end{frame}

	\section{Risultati pratici}

\begin{frame}
	\Large
	\sectionpage
	\centering
\end{frame}

\begin{frame}
	\frametitle{Color Coding}
	\centering
	\Large
	Tempi di esecuzione e memoria occupata
	
	\small
	\begin{figure}[h]
		\centering
		\begin{minipage}[ht]{.49\textwidth}
			\centering
			\begin{table}
				\centering
				\begin{tabular}{|c|c|c|c|}
					\hline
					\textsc{Dataset} & $q$  &               Tempo & Memoria \\ \hline \hline
					\textsc{NetInf}  & $13$ & \phantom{11}$0.39$s & \phantom{1}$11.20$MiB     \\ \hline
					\textsc{NetInf}  & $14$ & \phantom{11}$0.81$s & \phantom{1}$22.63$MiB     \\ \hline
					\textsc{NetInf}  & $15$ & \phantom{11}$1.66$s & \phantom{1}$45.21$MiB     \\ \hline
					\textsc{NetInf}  & $16$ & \phantom{11}$3.47$s & \phantom{1}$90.93$MiB     \\ \hline \hline
					\textsc{IMDb}    & $3$  & \phantom{1}$48.22$s & \phantom{1}$17.94$MiB     \\ \hline
					\textsc{IMDb}    & $4$  &           $105.94$s & \phantom{1}$34.91$MiB     \\ \hline
					\textsc{IMDb}    & $5$  &           $241.22$s & \phantom{1}$69.01$MiB     \\ \hline
					\textsc{IMDb}    & $6$  &           $557.48$s & $137.26$MiB     \\ \hline
				\end{tabular}
			\end{table}
		\end{minipage}
		%\pause
		\begin{minipage}[ht]{.49\textwidth}
			\centering
			\includegraphics[width=.9\textwidth]{images/3_color_coding}
			\caption{Scalabilità al variare dei cores usati}
		\end{minipage}
	\end{figure}
	
%	Dimensione tabella 
%	\begin{table}
%		\centering
%		\begin{tabular}{|c|c|c|}
%			\hline
%			\textsc{Dataset} & $q$  & Mem. \\ \hline \hline
%			\textsc{IMDb}    & $3$  &  $17.94$MiB  \\ \hline % 34.91  / 3.49
%			\textsc{IMDb}    & $4$  &  $34.91$MiB  \\ \hline % 34.91  / 3.49
%			\textsc{IMDb}    & $5$  &  $69.01$MiB  \\ \hline % 69.01  / 8.62
%			\textsc{IMDb}    & $6$  & $137.26$MiB  \\ \hline % 137.26 / 20.58
%		\end{tabular}
%	\end{table}
	
\end{frame}

\begin{frame}
	\frametitle{Query}
	\centering
	\begin{table}[ht]
		\centering
		\begin{tabular}{|c|c|c|c|c|c|c|c|}
			\cline{6-8}
			\multicolumn{5}{c|}{} & \multicolumn{3}{c|}{Tempi (in ms)} \\ \hline
			\textsc{Dataset} & $q$ & $|A|$ & $|B|$ & $r$      & \textsc{F-Count} 	& \textsc{F-Samp} & \textsc{Base} \\ \hline \hline
			\textsc{NetInf}  & $3$ & $100$ & $100$ & $1\,000$ & $20$             	& $4$               & $2$           \\ \hline
			\textsc{NetInf}  & $3$ & $100$ & $100$ & $5\,000$ & $60$             	& $30$              & $15$          \\ \hline \hline
			\textsc{NetInf}  & $5$ & $100$ & $100$ & $1\,000$ & $2\,682$         	& $426$             & $3$           \\ \hline
			\textsc{NetInf}  & $5$ & $100$ & $100$ & $5\,000$ & $4\,767$         	& $784$             & $20$          \\ \hline \hline
			\textsc{NetInf}  & $7$ & $100$ & $100$ & $100$    & $5\,455$ 			& $4$               & $2$           \\ \hline
			\textsc{NetInf}  & $7$ & $100$ & $100$ & $200$    & $16\,634$        	& $197$             & $2$           \\ \hline \hline
			\textsc{IMDb}    & $3$ & $10$  & $10$  & $100$    & $5\,035$         	& $66$              & $1$           \\ \hline
			\textsc{IMDb}    & $4$ & $10$  & $10$  & $100$    & $/$              	& $443$             & $8$           \\ \hline
			\textsc{IMDb}    & $5$ & $10$  & $10$  & $100$    & $/$              	& $781$             & $12$          \\ \hline
			\textsc{IMDb}    & $6$ & $10$  & $10$  & $100$    & $/$              	& $1\,379$          & $14$          \\ \hline
		\end{tabular}
		\medskip
		
		Tempi per il calcolo dell'indice di Bray-Curtis
		
		$r =$ Dimensione del campione
	\end{table}
\end{frame}

\definecolor{green}{RGB}{0,200,83}
\definecolor{orange}{RGB}{255, 171, 0}
\definecolor{red}{RGB}{213,0,0}

\begin{frame}
	\frametitle{$\epsilon$-approssimazione}
	\centering
	
	Confronto a parità di livello di approssimazione $\epsilon$
	\begin{table}[ht]
	%	\centering
		\begin{tabular}{|c|c|c|c|c|c|c|c|c|c|c|}
			\cline{3-11}
			\multicolumn{2}{c|}{} & \multicolumn{3}{c|}{\fcount} & \multicolumn{3}{c|}{\fsamp} & \multicolumn{3}{c|}{\base}\\
			\hline	
			$q$ & $\epsilon$ & $r$ & T    & VAR      & $r$ & T    & VAR      & $r$ & T   & VAR      \\ \hline
			$3$ & $0.20$     & \color{green}$2$  & $1$  & $0.0725$ & \color{orange}$400$     & $1$  & $0.1194$ & \color{   red}$420$     & $1$ & $0.1150$ \\ \hline
			$3$ & $0.10$     & \color{green}$3$  & $1$  & $0.0692$ & \color{   red}$1\,000$  & $1$  & $0.0601$ & \color{orange}$900$     & $1$ & $0.1338$ \\ \hline
			$3$ & $0.05$     & \color{green}$4$  & $1$  & $0.0535$ & \color{   red}$3\,200$  & $1$  & $0.0273$ & \color{orange}$1\,500$  & $1$ & $0.1025$ \\ \hline
			\hline
			$4$ & $0.20$     & \color{green}$3$  & $2$  & $0.0677$ & \color{   red}$1\,300$  & $1$  & $0.1194$ & \color{orange}$1\,300$  & $1$ & $0.2424$ \\ \hline
			$4$ & $0.10$     & \color{green}$5$  & $4$  & $0.0532$ & \color{   red}$3\,200$  & $2$  & $0.0992$ & \color{orange}$2\,500$  & $2$ & $0.1806$ \\ \hline
			$4$ & $0.05$     & \color{green}$10$ & $8$  & $0.0518$ & \color{   red}$8\,000$  & $4$  & $0.0612$ & \color{orange}$7\,900$  & $3$ & $0.1081$ \\ \hline
			\hline
			$5$ & $0.20$     & \color{green}$5$  & $6$  & $0.0511$ & \color{orange}$5\,000$  & $4$  & $0.1678$ & \color{   red}$6\,000$  & $3$ & $0.2234$ \\ \hline
			$5$ & $0.10$     & \color{green}$10$ & $18$ & $0.0370$ & \color{orange}$20\,000$ & $12$ & $0.0745$ & \color{   red}$30\,000$ & $8$ & $0.1234$ \\ \hline
			$5$ & $0.05$     & \color{green}$20$ & $58$ & $0.0204$ & \color{orange}$80\,000$ & $30$ & $0.0376$ & \color{   red}$/$       & $/$ & $/$      \\ \hline
		\end{tabular}
		
		\medskip
			Dati riferiti all'indice di Bray-Curtis su $\textsc{NetInf}$
		\small
		\begin{flushleft}
		
		
			Dimensione sottografi $|A| = |B| = 100$
		
			$r =$ Dimensione del campione
		
			$T = $ Tempo medio elaborazione (in millisecondi)
		
			$VAR = $ Varianza indici
		
		
		\end{flushleft}

	\end{table}
\end{frame}

\begin{frame}
	\frametitle{Nella pratica}
	\centering
	\begin{table}[h]
		\centering
		\begin{tabular}{c|c|l|l}
			Attore/Attrice & Attore/Attrice  & BC index & FJ index \\ 
			\hline
			Stan Laurel    & Oliver Hardy    & 0.936167 & 0.774053 \\
			Robert De Niro & Al Pacino       & 0.730935 & 0.231474 \\
			Woody Allen    & Meryl Streep    & 0.556071 & 0.222857 \\
			Meryl Streep   & Roberto Benigni & 0.482909 & 0.160181 \\
			%\hline
		\end{tabular}
		\medskip
		
		$\textsc{IMDb}$, Similarità tra ego-network di attori famosi (F-Samp)
	\end{table}


	\begin{table}[h]
		\centering
		\begin{tabular}{c|c|l|l}
			Sito           & Sito            & BC index & FJ index \\ 
			\hline
			cnn.com      & huffpost.com  & 0.936167 & 0.774053 \\
			nytimes.com  & cnn.com       & 0.730935 & 0.231474 \\
			huffpost.com & nytimes.com   & 0.556071 & 0.222857 \\
			%\hline
		\end{tabular}
		\medskip
		
		$\textsc{NetInf}$, Similarità tra siti di informazione (F-Samp)
	\end{table}

\end{frame}

\begin{frame}
	\frametitle{Conclusioni}
	\centering
	\begin{figure}[h]
		\begin{minipage}[t]{.32\textwidth}
			\centering
			\Large
			F-Count
			\medskip

			\small		
			\textbf{Pro}:
			\begin{itemize}
				\item Accurato anche con campioni di piccole dimensioni
				\item Varianza ridotta
			\end{itemize}
		
			\textbf{Contro}:
			\begin{itemize}
				\item Lento su grafi di elevate dimensioni
				\item Preprocessing grafo (una volta sola)
			\end{itemize}
		\end{minipage}\hfill
		%\pause
		\begin{minipage}[t]{.32\textwidth}
			\centering
			\Large
			F-Samp
			\medskip
			
			\small		
			\textbf{Pro}:
			\begin{itemize}
				\item Efficiente anche in grafi di elevate dimensioni
				\item Varianza ridotta
			\end{itemize}
			
			\textbf{Contro}:
			\begin{itemize}
				\item Necessita di campioni di grandi dimensioni
				\item Preprocessing grafo (una volta sola)
			\end{itemize}
		\end{minipage}\hfill
		%\pause
		\begin{minipage}[t]{.32\textwidth}
			\centering
			\Large
			Base
			\medskip
			
			\small		
			\textbf{Pro}:
			\begin{itemize}
				\item Efficiente anche in grafi di elevate dimensioni
%				\item 
			\end{itemize}
			
			\textbf{Contro}:
			\begin{itemize}
				\item Varianza elevata
				\item Necessita di campioni di grandi dimensioni
				\item Può non convergere al valore esatto
%				\item Preprocessing grafo (una 
			\end{itemize}
		\end{minipage}\hfill
		
	\end{figure}
\end{frame}

	
	% Fine
	\section{Fine}
	\begin{frame}
		\frametitle{Fine}
		\centering
		\Large
		Grazie per la Vostra attenzione
 	\end{frame}
 

\end{document}
