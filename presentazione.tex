\documentclass[11pt,svgnames,smaller]{beamer}

\usepackage{CommandsAndStyle}

\usepackage{listings}
\usepackage{amsthm}
\usepackage{amsmath}
\usepackage{subfig}
\usepackage{xcolor}
\usepackage{textpos}
\usepackage{transparent}
\usepackage{tikz}


\newcommand{\fsamp}{\textsc{F-samp}\xspace}
\newcommand{\base}{\textsc{base}\xspace}
\newcommand{\fcount}{\textsc{F-count}\xspace}
\mode<presentation>

\author{Gaspare Ferraro}

\author[Gaspare Ferraro]{\includegraphics[height=2cm]{images/cherubino}\\Gaspare Ferraro\\ \vspace{10pt} \small{Relatori\\ Roberto Grossi \\ Andrea Marino}}
\institute[Università di Pisa]{Università di Pisa}


\title{Similarità di sottografi nelle reti complesse}
\date{Pisa, 1 dicembre 2017}
\titlegraphic{\vfill}
% \includegraphics[height=0.4cm]{images/creative_commons.png}}

\setbeamercolor{title}{fg=black!65!black}

% \author[Relatore]{Roberto Grossi}

\begin{document}
	
	\begin{frame} 
	\titlepage
	\end{frame}
				
	\logo{\transparent{0.4}\includegraphics[height=2cm]{images/cherubino}}
		
%	\addtobeamertemplate{frametitle}{}{%
%		\begin{textblock*}{100mm}(\textwidth-1.5cm,0cm)
%			\includegraphics[height=1cm,width=1cm]{images/cherubino}
%		\end{textblock*}
%	}

	\section{Reti complesse}
	
	\begin{frame}
		\frametitle{Slide 1}
					
	\end{frame}


\section{Reti complesse}

	\begin{frame}
		
		\frametitle{Slide 1}
		
		\begin{table}[ht]
			\centering
			\begin{tabular}{|c|c|c|c|c|c|c|c|c|c|c|}
				\cline{3-11}
				\multicolumn{2}{c|}{} & \multicolumn{3}{c|}{\fcount} & \multicolumn{3}{c|}{\fsamp} & \multicolumn{3}{c|}{\base}\\
				\hline	
				$q$ & $\epsilon$ & $r$ & T    & VAR      & $r$ & T    & VAR      & $r$ & T   & VAR      \\ \hline
				$3$ & $0.20$     & $2$  & $1$  & $0.0725$ & $400$     & $1$  & $0.1194$ & $420$     & $1$ & $0.1150$ \\ \hline
				$3$ & $0.10$     & $3$  & $1$  & $0.0692$ & $1\,000$  & $1$  & $0.0601$ & $900$     & $1$ & $0.1338$ \\ \hline
				$3$ & $0.05$     & $4$  & $1$  & $0.0535$ & $3\,200$  & $1$  & $0.0273$ & $1\,500$  & $1$ & $0.1025$ \\ \hline
				\hline
				$4$ & $0.20$     & $3$  & $2$  & $0.0677$ & $1\,300$  & $1$  & $0.1194$ & $1\,300$  & $1$ & $0.2424$ \\ \hline
				$4$ & $0.10$     & $5$  & $4$  & $0.0532$ & $3\,200$  & $2$  & $0.0992$ & $2\,500$  & $2$ & $0.1806$ \\ \hline
				$4$ & $0.05$     & $10$ & $8$  & $0.0518$ & $8\,000$  & $4$  & $0.0612$ & $7\,900$  & $3$ & $0.1081$ \\ \hline
				\hline
				$5$ & $0.20$     & $5$  & $6$  & $0.0511$ & $5\,000$  & $4$  & $0.1678$ & $6\,000$  & $3$ & $0.2234$ \\ \hline
				$5$ & $0.10$     & $10$ & $18$ & $0.0370$ & $20\,000$ & $12$ & $0.0745$ & $30\,000$ & $8$ & $0.1234$ \\ \hline
				$5$ & $0.05$     & $20$ & $58$ & $0.0204$ & $80\,000$ & $30$ & $0.0376$ & $/$       & $/$ & $/$      \\ \hline
			\end{tabular}
			\caption{ Dataset NetInf, $|A| = |B| = 100$, $T = $ average running time, VAR $ = $ variance of solutions}
		\end{table}
	
	\end{frame}

	
	\section{Fine}
	
	\begin{frame}
		
		\frametitle{Fine}
		\centering
		\Large
		Grazie per la Vostra attenzione
			
 	\end{frame}
	%	\include{./tex/MST_Kruskal}

\end{document}
